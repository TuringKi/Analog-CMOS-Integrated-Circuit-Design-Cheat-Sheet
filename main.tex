%%%%%%%%%%%%%%%%%%%%%%%%%%%%%%%%%%%%%%%%%%%%%%%%%%%%%%%
% MatPlotLib and Random Cheat Sheet
%
% Edited by Michelle Cristina de Sousa Baltazar
%
% http://matplotlib.org/api/pyplot_summary.html
% http://matplotlib.org/users/pyplot_tutorial.html
%
%%%%%%%%%%%%%%%%%%%%%%%%%%%%%%%%%%%%%%%%%%%%%%%%%%%%%%%

\documentclass{article}
\usepackage[landscape]{geometry}
\usepackage{url}
\usepackage{multicol}
\usepackage{amsmath}
\usepackage{amsfonts}
\usepackage{tikz}
\usetikzlibrary{decorations.pathmorphing}
\usepackage{amsmath,amssymb}

\usepackage{colortbl}
\usepackage{xcolor}
\usepackage{mathtools}
\usepackage{amsmath,amssymb}
\usepackage{enumitem}
\usepackage{circuitikzgit}
\usetikzlibrary{fit, shapes, arrows, patterns, decorations.text, decorations.markings}


\usepackage{geometry}
 \geometry{
  a2paper
 }


\usepackage[utf8]{inputenc}

\advance\topmargin-.8in
\advance\textheight3in
\advance\textwidth3in
\advance\oddsidemargin-1.5in
\advance\evensidemargin-1.5in
\parindent0pt
\parskip2pt
\newcommand{\hr}{\centerline{\rule{3.5in}{1pt}}}
%\colorbox[HTML]{e4e4e4}{\makebox[\textwidth-2\fboxsep][l]{texto}
\begin{document}

\begin{center}{\huge{\textbf{Analog CMOS Integrated Circuit Design Cheat Sheet}}}\\
{\large By Xiao Ma (mxlol233@outlook.com)[https://github.com/TuringKi/Analog-CMOS-Integrated-Circuit-Design-Cheat-Sheet]}
\end{center}
\begin{multicols*}{3}

\tikzstyle{mybox} = [draw=black, fill=white, very thick,
    rectangle, rounded corners, inner sep=10pt, inner ysep=10pt]
\tikzstyle{fancytitle} =[fill=black, text=white, font=\bfseries]
%------------ CONTEÚDO CAIXA RANDOM ---------------
\begin{tikzpicture}
\node [mybox] (box){%
    \begin{minipage}{0.3\textwidth}
		Process parameters ($n$,$V_{TH}$,$KP$, $V_E$):
      \begin{equation}
        t_{OX} = \frac{L_{min}}{50}
       \end{equation}
       \begin{equation}
        t_{si} = \sqrt{\frac{2\epsilon_{si}(\Phi - V_{BD})}{qN_B}}
       \end{equation}
       \begin{equation}
        C_{OX} = \frac{\bold{\epsilon}_{OX}}{t_{OX}}
       \end{equation}
       \begin{equation}
        C_{D} = \frac{\bold{\epsilon}_{si}}{t_{si}}
      \end{equation}
       \begin{equation}
        KP = \mu C_{OX}
       \end{equation}
      \begin{equation}
       \bold{\beta} = KP\frac{W}{L}
      \end{equation}
\begin{equation}
  Q_{dep} = \sqrt{4q\epsilon_{si}|\Phi_F|N_{sub}}
\end{equation}
\begin{equation}
  V_{TH0} = \Phi_{MS} + 2 \Phi_F + \frac{Q_{dep}}{C_{OX}}
\end{equation}
Changing the Gate voltage $V_{GS}$ will thus
change the conductivity of the channel and hence the current $I_{DS}$. In a similar way, changing
the Bulk voltage $V_{BS}$ will thus also change the conductivity of the channel and will thus change the
current $I_{DS}$ as well. The  gate gives the MOST operation, whereas the bulk gives JFET
operation. 

Indeed, a Junction FET is by definition a FET in which the current is controlled by
a junction capacitance.


All MOST devices are thus parallel combinations of MOSTs and JFETs.

The Bulk voltage works like a "back-gate", which is also called as "body effect":
      \begin{equation}
        V_{TH} = V_{TH0} + \gamma (\sqrt{|2\Phi_F| + V_{BS}} - \sqrt{|2\Phi_F|})
       \end{equation}
       \begin{equation}
        n = \frac{\gamma}{\sqrt{|2\Phi_F| + V_{BS}}} = 1 + \frac{C_D}{C_{OX}}
       \end{equation}
In linear region:
\begin{equation}
  I_{DS} = \beta[(V_{GS}-V_{TH})V_{DS} - \frac{1}{2}V_{DS}^2]
\end{equation}
\begin{equation}
  R_{on} = \frac{1}{\beta (V_{GS} - V_{TH})}
\end{equation}

      Channel-Length modulation in saturation region:
      \begin{equation}
        K' = \frac{KP}{2n}
      \end{equation}
      \begin{equation}
        \lambda = \frac{1}{V_E L}
      \end{equation}
      \begin{equation}
       I_{DS} =  K' \frac{W}{L}(V_{GS}-V_{TH})^2(1+\lambda V_{DS})
      \end{equation}
      \begin{equation}
        r_{o} = \frac{\partial V_{DS}}{\partial I_{DS}} \approx \frac{1}{\lambda I_{DS}} = \frac{V_E L}{I_{DS}}
      \end{equation}
Saturation region has three distinctive regions: weak-inversion (exponential region),  
strong-inversion, and velocity saturation.
 


    \end{minipage}
};
%------------ CAIXA RANDOM ---------------------
\node[fancytitle, right=10pt] at (box.north west) {Model of MOS Transistors};
\end{tikzpicture}

%------------ CONTEUDO EXEMPLO BASICO ---------------------
\begin{tikzpicture}
  \node [mybox] (box){%
      \begin{minipage}{0.3\textwidth}
\begin{center}
        \small{\begin{tabular}{lp{1.4cm} l}
          {\bf Names} & {\bf Symbols} & {\bf Values} \\
          dielectric constant of sub-silicon & $\epsilon_{si}$ & $1$ pF/cm \\ \hline
          dielectric constant of gate-oxide & $\epsilon_{OX}$ & $0.34$ pF/cm \\ \hline
          electron charge & $q$ & $1.6 \times 10^{-19}$ C \\ \hline
        minium channel length & $L_{min}$ & $0.35$ $\mu$m \\ \hline
        width of gate-oxide  & $t_{OX}$ & $0.1$ nm \\ \hline
        width of depletion layer & $t_{si}$ & $7$ nm \\ \hline
        junction built-in voltage & $\Phi$ & $0.6$ V\\ \hline
        drain-bulk voltage & $V_{BD}$ & $-3.3 $V \\ \hline
         gate-oxide capacitance & $C_{OX}$ & $0.5$ $\mu$F/cm$^2$ \\ \hline
         depletion layer capacitance & $C_{D}$ & $0.1$ $\mu$F/cm$^2$ \\ \hline
         bulk doping level & $N_B$ & $4\times10^{17}$ cm$^{-3}$ \\ \hline
         P type mobility rate & $\mu_{p}$& $250$ cm$^2$/Vs\\ \hline
         N type mobility rate & $\mu_{n}$&  $600$ cm$^2$/Vs \\ \hline
         N type KP & $KP_{n}$ &  $300$ $\mu$A/V$^2$ \\ \hline
         & $n$ &  $1.2 \cdots 1.5$ \\ \hline
         & $|2\Phi_F|$ &  $0.6$ V \\ \hline
         & $\gamma$ &  $0.5 \cdots 0.8$ V$^{\frac{1}{2}}$ \\ \hline
         N type $K'$ & $K'_{n}$ &  $100$ $\mu$A/V$^2$ \\ \hline
         P type $K'$ & $K'_{p}$ &  $40$ $\mu$A/V$^2$ \\ \hline
           & $V_{GSTt}$ &  $70$ mV \\ \hline
           & $v_{sat}$ &  $10^{7}$ cm/s \\ \hline
           & $\theta L$ &  $0.2$ $\mu$m/V \\ \hline
           
            \end{tabular}}

            
          \end{center}

            \end{minipage}
  };
  %------------ EXEMPLO BASICO BOX ---------------------
  \node[fancytitle, right=10pt] at (box.north west) {Value Examples In $0.35\mu m$ Process Nodes };
  \end{tikzpicture}


  \begin{tikzpicture}
    \node [mybox] (box){%
        \begin{minipage}{0.3\textwidth}

         \begin{equation}
          I_{DS}= I_{D0}\frac{W}{L}e^{\frac{V_{GS}}{n\frac{KT}{q}}}
         \end{equation}

  
\begin{equation}
g_{m} = \frac{\partial I_{DS}}{\partial V_{GS}} = \frac{I_{DS}}{n\frac{KT}{q}}
\end{equation}



   
  
              \end{minipage}
    };
    \node[fancytitle, right=10pt] at (box.north west) {Weak-Inversion};
    \end{tikzpicture}

    \begin{tikzpicture}
      \node [mybox] (box){%
          \begin{minipage}{0.3\textwidth}
Ignore channel-length modulation:
            \begin{equation}
              I_{DS} =  K' \frac{W}{L}(V_{GS}-V_{TH})^2
             \end{equation}

             \begin{equation}
              g_m = \frac{2I_{DS}}{V_{GS} - V_{TH}}
             \end{equation}
         

                \end{minipage}
      };
      %------------ EXEMPLO BASICO BOX ---------------------
      \node[fancytitle, right=10pt] at (box.north west) {strong-inversion};
      \end{tikzpicture}


    
        \begin{tikzpicture}
          \node [mybox] (box){%
              \begin{minipage}{0.3\textwidth}
                The voltage  and current  at transition point between weak-inversion and strong-inversion:
                \begin{equation}
                 V_{GSt} =  2n\frac{KT}{q} + V_{TH} 
                \end{equation}
                \begin{equation}
                 I_{DSt} \approx  K'\frac{W}{L} (2n\frac{KT}{q})^2  
                \end{equation}
          EKV model, a smooth model for weak-inversion and strong-inversion regions:
   
          \begin{equation}
           I_{DS} =  K' \frac{W}{L}(V_{GS}-V_{TH})^2[ ln(1+e^{\frac{V_{GS}}{V_{GSt}}}) ]^2
          \end{equation}
          Let:
          \begin{equation}
           v =  \frac{V_{GS}}{V_{GSt}}
          \end{equation}
          \begin{equation}
           i =  \frac{I_{DS}}{I_{DSt}} = [ ln(1+e^{v}) ]^2
          \end{equation}
          then,
          \begin{equation}
            v  = ln(e^{\sqrt{i}} -1)
           \end{equation}
           \begin{equation}
            V_{GS} - V_{TH}  = V_{GSTt}ln(e^{\sqrt{i}} -1)
           \end{equation}
where:
\begin{equation}
  V_{GSTt} = V_{GSt} - V_{TH} = 2n\frac{KT}{q} 
\end{equation}
          When $v=1$, $i=1$, we also have:
          \begin{equation}
           I_{DSt} = K'\frac{W}{L}(V_{GSt} - V_{TH})^2
          \end{equation}
        
                    \end{minipage}
          };
          %------------ EXEMPLO BASICO BOX ---------------------
          \node[fancytitle, right=10pt] at (box.north west) {Transition Point Between Weak-Inversion and Strong-Inversion};
          \end{tikzpicture}
    

              
        \begin{tikzpicture}
          \node [mybox] (box){%
              \begin{minipage}{0.3\textwidth}
    
                \begin{equation}
                  I_{DS} =  WC_{OX}v_{sat}(V_{GS}-V_{TH})
                 \end{equation}
                 \begin{equation}
                  g_m=  WC_{OX}v_{sat}
                 \end{equation}

        
                    \end{minipage}
          };
          %------------ EXEMPLO BASICO BOX ---------------------
          \node[fancytitle, right=10pt] at (box.north west) {Velocity Saturation};
          \end{tikzpicture}
    

              
        \begin{tikzpicture}
          \node [mybox] (box){%
              \begin{minipage}{0.3\textwidth}

               

                A smooth model for strong-inversion  and velocity saturation regions:
                \begin{equation}
                  I_{DS} = \frac{K' \frac{W}{L}(V_{GS}-V_{TH})^2}{1 + \theta (V_{GS} - V_{TH})} 
                 \end{equation}
        
                 where:
        \begin{equation}
          \theta = \frac{\mu}{2n}\frac{1}{v_{sat}L}
         \end{equation}
         $\theta L$ is constant:
         \begin{equation}
          \theta L = \frac{\mu}{2n}\frac{1}{v_{sat}}
         \end{equation}

         \begin{equation}
          g_{m,sat} =WC_{OX}v_{sat}= \frac{K'W}{\theta L}
         \end{equation}


         The voltage  and current  at transition point between strong-inversion  and velocity saturation:
         \begin{equation}
          V_{GSt} = \frac{1}{\theta} + V_{TH} = 2nL\frac{v_{sat}}{\mu} + V_{TH}
         \end{equation}
         \begin{equation}
          I_{DSt} = K'WL(2n\frac{v_{sat}}{\mu})^2
         \end{equation}


                    \end{minipage}
          };
          %------------ EXEMPLO BASICO BOX ---------------------
          \node[fancytitle, right=10pt] at (box.north west) {Transition Point Between Strong-Inversion and Velocity Saturation};
          \end{tikzpicture}
   
          
    

          \begin{tikzpicture}


            
            \node [mybox] (box){%
                \begin{minipage}{0.3\textwidth}
      
                  A system with $N$ frequency-dependent elements,  th transfer function in complex-frequency form:
                  \begin{equation}
                    H_s =  \frac{ a_0 + a_1s }{ 1+ b_1s     }
                  \end{equation}  
where:
                  \begin{equation}
                    a_0 = H^0
                   \end{equation}  
                   \begin{equation}
                    b_1 = \sum_{i=1}^{N}\tau_i^0
                   \end{equation}  
                   \begin{equation}
                    a_1 = \sum_{i=1}^{N}\tau_i^0 H^i
                   \end{equation}
                   where $H^0$ is the low-frequency gain:
                   \begin{equation}
                    H^0: C_1,C_2,\cdots,C_N = 0 
                     \end{equation}
                     or:
                     \begin{equation}
                      H^0: L_1,L_2,\cdots,L_N = 0 
                       \end{equation}
                   For capacitor:
\begin{equation}
  \tau_i^0 = C_i R_i^0 
 \end{equation}
 or for inductor:
 \begin{equation}
  \tau_i^0 = \frac{L_i}{R_i^0}
 \end{equation}    
 where $R_i^0$ is the resistance seen by the capacitor $C_i$ looking into port i with
 all other reactive elements connected to the other ports at their zero value
 (hence the superscript), namely open-circuited capacitors (and short circuited
 inductors):
 \begin{equation}
R_i^0:  C_1,C_2,C_{i -1},C_{i + 1}, \cdots, C_N = 0
 \end{equation}
 or:
 \begin{equation}
  R_i^0: L_1,L_2,\cdots,L_{i - 1}, L_{i + 1}, \cdots, L_N = 0
   \end{equation}
                
      Bandwidth estimation by 1st-order TTCs:
      \begin{equation}
        \omega_h \approx \frac{1}{b_1 - \frac{a_1}{a_0}} = \frac{1}{\sum_{i=1}^{N}\tau_i^0(1 - \frac{H^i}{H^0})}
       \end{equation}  

                      \end{minipage}
            };
            %------------ EXEMPLO BASICO BOX ---------------------
            \node[fancytitle, right=10pt] at (box.north west) {1st Order Time-constants and Transfer-Constants (TTCs)};
            \end{tikzpicture}



      

          \begin{tikzpicture}


            
            \node [mybox] (box){%
                \begin{minipage}{0.3\textwidth}
      
                  A system with $2$ frequency-dependent elements,  th transfer function:
                  \begin{equation}
                    H_s =  \frac{ a_0 + a_1s  +a_2s^2 }{  1+b_1s + b_2s^2     }
                  \end{equation}  
where:
                  \begin{equation}
                    a_0 = H^0
                   \end{equation}  
                   \begin{equation}
                    b_1 = \tau_1^0 + \tau_2^0
                   \end{equation}  
                   \begin{equation}
                    a_1 = \tau_1^0 H^1 + \tau_2^0 H^2
                   \end{equation}  
                   \begin{equation}
                    b_2 = \tau_1^0  \tau_2^1 = \tau_2^0 \tau_1^2 
                   \end{equation}
                   \begin{equation}
                    a_2 = \tau_1^0  \tau_2^1 H^{12} = \tau_2^0 \tau_1^2  H^{12}
                   \end{equation}
in which, $H^1$ ($H^2$) evaluated with the frequency-dependent element at the port $1$($2$) at its infinite value (i.e., shorted capacitors and
open inductors):
\begin{equation}
  H^1: C_1=\infty, C_2 = 0
\end{equation}
\begin{equation}
  H^2: C_1=0, C_2 = \infty
\end{equation}
or:
\begin{equation}
  H^1: L_1=\infty, L_2 = 0
\end{equation}
\begin{equation}
  H^2: L_1=0, L_2 = \infty
\end{equation}

                   For capacitor:
\begin{equation}
  \tau_1^0 = C_1 R_1^0 
\end{equation}
\begin{equation}
  \tau_2^0 = C_2 R_2^0
\end{equation}
\begin{equation}
  \tau_1^2 = C_1 R_1^{2} 
\end{equation} 
\begin{equation}
  \tau_2^1 = C_2 R_2^{1}  
 \end{equation}
 or for inductor:
 \begin{equation}
  \tau_1^0 = \frac{L_1} { R_1^0} 
\end{equation}
\begin{equation}
  \tau_2^0 =  \frac{L_2} {R_2^0}
\end{equation}
\begin{equation}
  \tau_1^2 =  \frac{L_1} { R_1^{2} }
\end{equation} 
\begin{equation}
  \tau_2^1 =  \frac{L_2} {R_2^{1}  }
 \end{equation}
 in which, $R_1^2$ is the resistance seen by $C_1$ (the subscript) when $C_2$ (the superscript)
 is infinite valued (shorted):
 \begin{equation}
  R_{1}^{2} : C_2 = \infty
 \end{equation}
 \begin{equation}
  R_{2}^{1} : C_1 = \infty
 \end{equation}
   or:
   \begin{equation}
    R_{1}^{2} : L_2 = \infty
   \end{equation}
   \begin{equation}
    R_{2}^{1} : L_1 = \infty
   \end{equation}                   
\end{minipage}
            };
            %------------ EXEMPLO BASICO BOX ---------------------
            \node[fancytitle, right=10pt] at (box.north west) {2nd Order Time-constants and Transfer-Constants (TTCs)};
            \end{tikzpicture}


 
          \begin{tikzpicture}


            
            \node [mybox] (box){%
                \begin{minipage}{0.3\textwidth}

                A system with $N$ frequency-dependent elements,  th transfer function:
                \begin{equation}
                  H_s =  \frac{ a_0 + a_1s + a_2s^2  \cdots +a_ns^n +\cdots  }{1+ b_1s  + b_2s^2 \cdots +b_ns^n +\cdots     }
                \end{equation}  
           where:
           \begin{equation}
           a_0 = H^0
          \end{equation}  
          \begin{equation}
            b_1 = \sum_{i=1}^{N}\tau_i^0
           \end{equation}  
           \begin{equation}
            a_1 = \sum_{i=1}^{N}\tau_i^0 H^i
           \end{equation}
           \begin{equation}
            b_2 = \sum_{i=1}^{i< j} \sum_{j=i+1}^{j\leq N} \tau_i^0 \tau_j^i
           \end{equation}  
           \begin{equation}
            a_2 = \sum_{i=1}^{i< j} \sum_{j=i+1}^{j\leq N} \tau_i^0 \tau_j^i H^{ij}
           \end{equation}
           \begin{equation}
            b_n = \sum_{i=1}^{i< j} \sum_{j=i+1}^{j<k} \sum_{k=j+1\cdots}^{k<...\leq N} \tau_i^0 \tau_j^i \tau_k^{ij} \cdots
           \end{equation}  
           \begin{equation}
            a_n = \sum_{i=1}^{i< j} \sum_{j=i+1}^{j<k} \sum_{k=j+1\cdots}^{k<...\leq N} \tau_i^0 \tau_j^i \tau_k^{ij} \cdots H^{ijk...}
           \end{equation}  
        

and for capacitor:
\begin{equation}
  \tau_i^0 = C_i R_i^0 
 \end{equation}
 \begin{equation}
  \tau_i^{jk\cdots} =C_i R_i^{jk\cdots} 
 \end{equation} 
 or for inductor:
 \begin{equation}
  \tau_i^0 = \frac{L_i}{R_i^0}
 \end{equation}    
 \begin{equation}
  \tau_i^{jk\cdots} =\frac{L_i} {R_i^{jk\cdots}} 
 \end{equation} 
                In the form of zeros and poles:
                \begin{equation}
                  H_s = a_0 \frac{ (1-\frac{s}{z_1}) (1-\frac{s}{z_1}) \cdots (1-\frac{s}{z_m})   }{  (1-\frac{s}{p_1}) (1-\frac{s}{p_1}) \cdots (1-\frac{s}{p_n})    }
                \end{equation}  
  
                      \end{minipage}
            };
            %------------ EXEMPLO BASICO BOX ---------------------
            \node[fancytitle, right=10pt] at (box.north west) {General Time-constants and Transfer-Constants (TTCs)};
            \end{tikzpicture}


          \begin{tikzpicture}


            
            \node [mybox] (box){%
                \begin{minipage}{0.3\textwidth}
                  Common Source, with resistance $R_D$:
  \begin{center}

                  \begin{tikzpicture}[scale=1.00, transform shape, /tikz/circuitikz/bipoles/length=1.50cm, american currents, american voltages, voltage dir=RP]
                   
                    \coordinate (D) at (4.1,2.4);
                    \coordinate (G) at (3,1.4);
                    \coordinate (S) at (4.1,0.4);
                    \coordinate (out) at (5.1,2.4);
                    \coordinate (V_{out}) at (6.1,2.4);
                    \coordinate (in) at (2,1.4);
                    \coordinate (V_{in}) at (0,1.4);
                    \coordinate (1) at (4.1,4.4);
                    \coordinate (0) at (4.1,0);
                    \coordinate (2) at (4.1,4.42);
                   \ctikzset{tripoles/mos style/arrows}
                    \draw (4.1,1.4) node[nmos, , xscale=1.0, yscale=1.0, rotate=0] (M1) {$M_{1}$};
                    \draw (M1.D) -- (D) (M1.G) -- (G) (M1.S) -- (S);
                    \draw[-] (out) to (V_{out});
                    \draw (in) to [R, bipoles/length=0.75cm, l_={${R_1}$}, n=R1] (V_{in});
                    \draw[-] (D) to (out);
                    \draw (D) to [R, bipoles/length=0.90cm, l_={${R_D}$}, n=RD] (1);
                    \draw[-] (S) to (0);
                    \draw[-] (1) to (2);
                    \draw[-] (G) to (in);
                    \draw (D) node[circ] {};
                    \draw (G) node[circ] {};
                    \draw (S) node[circ] {};
                    \draw (V_{out}) node[ocirc] {};
                    \draw (V_{in}) node[ocirc] {};
                    \draw (0) node[sground] {};
                    \draw (2) node[rground, rotate=180] {};
                    \draw[anchor=west] (4.3,4.92) node {$V_{\mathrm{DD}}$};
                    \draw[anchor=south east] (out) node {out};
                    \draw[anchor=south east] (V_{out}) node {V$_{\mathrm{{out}}}$};
                    \draw[anchor=south east] (in) node {in};
                    \draw[anchor=south east] (V_{in}) node {V$_{\mathrm{{in}}}$};


                  \end{tikzpicture}
               
  \end{center} 
  Transfer function of voltage gain in low-frequency:
      \begin{equation}
        A_{v}^0 =\frac{V_{out}}{V_{in}}= -g_mR_{D}
      \end{equation}
      consider channel-length modulation:
      \begin{equation}
        A_{v}^0  =\frac{V_{out}}{V_{in}}= -g_mR_{D} || r_o = -g_m\frac{R_D r_o}{R_D + r_o}
      \end{equation}
      The equivalent resistance at point $out$ look down $M1$:
      \begin{equation}
        R_{out} = r_o
      \end{equation}
  Consider high-frequency gain with $3$ capacitors $c_{\pi}, c_{\mu}, c_{L}$:

    \begin{center} 
      \begin{tikzpicture}[scale=1.00, transform shape, /tikz/circuitikz/bipoles/length=1.50cm, american currents, american voltages, voltage dir=RP]
   
        \coordinate (D) at (4.1,2.4);
        \coordinate (G) at (3,1.4);
        \coordinate (S) at (4.1,0.4);
        \coordinate (out) at (5.1,2.4);
        \coordinate (V_{out}) at (6.1,2.4);
        \coordinate (in) at (2,1.4);
        \coordinate (V_{in}) at (0,1.4);
        \coordinate (1) at (4.1,4.4);
        \coordinate (0) at (4.1,0);
        \coordinate (2) at (4.1,4.42);
        \coordinate (3) at (2,0.4);
        \coordinate (5) at (2,2.4);
        \coordinate (7) at (3.05,2.4);
        \coordinate (4) at (2,-0);
        \coordinate (10) at (5.1,1.4);
        \coordinate (11) at (5.1,1);
       \ctikzset{tripoles/mos style/arrows}
        \draw (4.1,1.4) node[nmos, , xscale=1.0, yscale=1.0, rotate=0] (M1) {$M_{1}$};
        \draw (M1.D) -- (D) (M1.G) -- (G) (M1.S) -- (S);
        \draw[-] (out) to (V_{out});
        \draw (in) to [R, bipoles/length=0.75cm, l_={${R_1}$}, n=R1] (V_{in});
        \draw[-] (D) to (out);
        \draw (D) to [R, bipoles/length=0.90cm, l_={${R_D}$}, n=RD] (1);
        \draw[-] (S) to (0);
        \draw[-] (1) to (2);
        \draw[-] (G) to (in);
        \draw[color=red] (in) to [C, bipoles/length=0.90cm, l_={${C_{\pi}}$}, n=C0] (3);
        \draw[-] (in) to (5);
        \draw[color=red] (5) to [C, bipoles/length=0.90cm, l_={${C_{\mu}}$}, n=C1] (7);
        \draw[-] (7) to (D);
        \draw[-] (3) to (4);
        \draw[color=red] (out) to [C, bipoles/length=0.90cm, l_={${C_{L}}$}, n=C2] (10);
        \draw[-] (10) to (11);
        \draw (D) node[circ] {};
        \draw (G) node[circ] {};
        \draw (S) node[circ] {};
        \draw (out) node[circ] {};
        \draw (V_{out}) node[ocirc] {};
        \draw (in) node[circ] {};
        \draw (V_{in}) node[ocirc] {};
        \draw (0) node[sground] {};
        \draw (2) node[rground, rotate=180] {};
        \draw[anchor=west] (4.3,4.92) node {$V_{\mathrm{DD}}$};
        \draw (4) node[sground] {};
        \draw (11) node[sground] {};
        \draw[anchor=south east] (out) node {out};
        \draw[anchor=south east] (V_{out}) node {V$_{\mathrm{{out}}}$};
        \draw[anchor=south east] (in) node {in};
        \draw[anchor=south east] (V_{in}) node {V$_{\mathrm{{in}}}$};

      \end{tikzpicture}
    \end{center} 

    \begin{equation}
      \tau_{\pi}^0 =  C_{\pi} R_1 
    \end{equation}
    \begin{equation}
      \tau_{\mu}^0 =  C_{\mu} (R_{left} + R_{right} + G_m R_{left} R_{right} )=C_{\mu} (R_1 + R_D + g_m R_1 R_D )
    \end{equation}
    \begin{equation}
      \tau_{L}^{0} =  C_{L} R_D
    \end{equation}
    \begin{equation}
      \tau_{\mu}^{\pi} =  C_{\mu} R_D
    \end{equation}
    \begin{equation}
      \tau_{\pi}^{L} =  C_{\pi} R_1
    \end{equation}
    \begin{equation}
      \tau_{\mu}^{L} =  C_{\mu} R_1
    \end{equation}
    \begin{equation}
      \tau_{\mu}^{\pi L} = \tau_{L}^{\pi \mu} =  0
    \end{equation}
    \begin{equation}
      H^{\pi} = H^{L} = 0
    \end{equation}
    \begin{equation}
      H^{\mu} = \frac{r_m || R_D}{R_1 + r_m || R_D} = \frac{R_D}{R_1 + R_D + g_m R_1 R_D}
    \end{equation}
    \begin{equation}
      H^{\pi \mu} =  H^{L \mu} =H^{L \pi} =0
    \end{equation}
    \begin{equation}
     a_0 = A_{v}^0
    \end{equation}
     \begin{equation}
      b_1 = \tau_{\pi}^{0} + \tau_{\mu}^{0} + \tau_{L}^{0} = R_1[C_{\pi} + C_{\mu}(1+g_m R_D)] + R_D(C_{\mu} + C_L)
     \end{equation}
     \begin{equation}
      a_1 = \tau_{\pi}^{0} H^{\pi}+ \tau_{\mu}^{0} H^{\mu} + \tau_{L}^{0}H^{L} = C_{\mu} R_D
     \end{equation}
     \begin{equation}
      b_2 = \tau_{\pi}^{0} \tau_{\mu}^{\pi} + \tau_{L}^{0} \tau_{\mu}^{L} +\tau_{L}^0 \tau_{\pi}^{L} = R_1 R_D(C_{\mu}C_{\pi} + C_{\mu}C_{L} + C_{\pi}C_L) = R_{left}R_{right} (\Delta C)^2
     \end{equation}
     \begin{equation}
      a_2 = \tau_{\pi}^{0} \tau_{\mu}^{\pi} H^{\pi \mu}+ \tau_{L}^{0} \tau_{\mu}^{L} H^{L \mu} +\tau_{L}^0 \tau_{\pi}^{L} H^{L \pi} = 0
     \end{equation}
     \begin{equation}
      b_3 = \tau_{\pi}^{0} \tau_{\mu}^{\pi} \tau_{L}^{\pi \mu} = 0
     \end{equation}
     \begin{equation}
      a_3 = \tau_{\pi}^{0} \tau_{\mu}^{\pi} \tau_{L}^{\pi \mu} H^{\pi \mu L} = 0
     \end{equation}
     Finally, we have:
     \begin{equation}
      A_{v}(s) = \frac{A_v^0(1 - r_m C_{\mu} s)}{ 1+[R_1[C_{\pi} + C_{\mu}(1+g_m R_D)] + R_D(C_{\mu} + C_L)]s +R_1 R_D(C_{\mu}C_{\pi} + C_{\mu}C_{L} + C_{\pi}C_L)s^2 }
     \end{equation}
                      \end{minipage}
            };
            %------------ EXEMPLO BASICO BOX ---------------------
            \node[fancytitle, right=10pt] at (box.north west) {Single Stage Amplifier: Common Source (CS), with Resistive Load};
            \end{tikzpicture}
  
            \begin{tikzpicture}


            
              \node [mybox] (box){%
                  \begin{minipage}{0.3\textwidth}
                    Common Source, with resistance $R_D$ and $R_S$:
\begin{center}

  \begin{tikzpicture}[scale=1.00, transform shape, /tikz/circuitikz/bipoles/length=1.50cm, american currents, american voltages, voltage dir=RP]
  

    \coordinate (D) at (4.1,3.4);
    \coordinate (G) at (3,2.4);
    \coordinate (S) at (4.1,1.4);
    \coordinate (out) at (5.1,3.4);
    \coordinate (V_{out}) at (6.1,3.4);
    \coordinate (in) at (2,2.4);
    \coordinate (V_{in}) at (0,2.4);
    \coordinate (1) at (4.1,5.4);
    \coordinate (3) at (4.1,0.4);
    \coordinate (0) at (4.1,0);
    \coordinate (2) at (4.1,5.42);
   \ctikzset{tripoles/mos style/arrows}
    \draw (4.1,2.4) node[nmos, , xscale=1.0, yscale=1.0, rotate=0] (M1) {$M_{1}$};
    \draw (M1.D) -- (D) (M1.G) -- (G) (M1.S) -- (S);
    \draw[-] (out) to (V_{out});
    \draw (in) to [R, bipoles/length=0.75cm, l_={${R_1}$}, n=R1] (V_{in});
    \draw[-] (D) to (out);
    \draw (D) to [R, bipoles/length=0.90cm, l_={${R_D}$}, n=RD] (1);
    \draw (S) to [R, bipoles/length=0.75cm, l_={${R_S}$}, n=RS] (3);
    \draw[-] (3) to (0);
    \draw[-] (1) to (2);
    \draw[-] (G) to (in);
    \draw (D) node[circ] {};
    \draw (G) node[circ] {};
    \draw (S) node[circ] {};
    \draw (V_{out}) node[ocirc] {};
    \draw (V_{in}) node[ocirc] {};
    \draw (0) node[sground] {};
    \draw (2) node[rground, rotate=180] {};
    \draw[anchor=west] (4.3,5.92) node {$V_{\mathrm{DD}}$};
    \draw[anchor=south east] (out) node {out};
    \draw[anchor=south east] (V_{out}) node {V$_{\mathrm{{out}}}$};
    \draw[anchor=south east] (in) node {in};
    \draw[anchor=south east] (V_{in}) node {V$_{\mathrm{{in}}}$};

  \end{tikzpicture}


                  \end{center}
                  The low frequency gain:
  \begin{equation}
    A_{v}^{0} = -\frac{g_m R_D}{1 + g_m R_S}
  \end{equation}
  The equivalent transconductance:
  \begin{equation}
    G_m = \frac{g_m}{1 + g_m R_S}
  \end{equation}
  consider channel-length modulation and body effect:
  \begin{equation}
    G_m = \frac{g_m r_o}{R_S + [1 + (g_m + g_{mb})R_s]r_o}
  \end{equation}
  \begin{equation}
    R_{out} = r_o + [1 + (g_m + g_{mb})r_o]R_S \approx g_m r_o R_S
  \end{equation}
  \begin{equation}
    A_v^0 = -G_m(R_{out} || R_D) = - \frac{g_m r_o}{R_S + [1 + (g_m + g_{mb})R_s]r_o} \{[r_o + [1 + (g_m + g_{mb})r_o]R_S] || R_D \} 
  \end{equation}
   
                        \end{minipage}
              };
              %------------ EXEMPLO BASICO BOX ---------------------
              \node[fancytitle, right=10pt] at (box.north west) {Single Stage Amplifier: Common Source (CS), with Source Degeneration};
              \end{tikzpicture}

              \begin{tikzpicture}


            
                \node [mybox] (box){%
                    \begin{minipage}{0.3\textwidth}
  
  
                      Common Source, with diode-connected transistor $M2$:
  
                      \begin{center}

    \begin{tikzpicture}[scale=1.00, transform shape, /tikz/circuitikz/bipoles/length=1.50cm, american currents, american voltages, voltage dir=RP]
      \coordinate (D) at (4.1,2.4);
      \coordinate (G) at (3,1.4);
      \coordinate (S) at (4.1,0.4);
      \coordinate (out) at (5.1,2.4);
      \coordinate (V_{out}) at (6.1,2.4);
      \coordinate (in) at (2,1.4);
      \coordinate (V_{in}) at (0,1.4);
      \coordinate (S2) at (4.1,2.8);
      \coordinate (G2) at (3,3.8);
      \coordinate (3) at (2.6,3.8);
      \coordinate (4) at (2.6,5.2);
      \coordinate (5) at (4.1,5.2);
      \coordinate (D2) at (4.1,4.8);
      \coordinate (0) at (4.1,0);
      \coordinate (2) at (4.1,5.6);
     \ctikzset{tripoles/mos style/arrows}
      \draw (4.1,1.4) node[nmos, , xscale=1.0, yscale=1.0, rotate=0] (M1) {$M_{1}$};
      \draw (M1.D) -- (D) (M1.G) -- (G) (M1.S) -- (S);
      \draw[-] (out) to (V_{out});
      \draw (in) to [R, bipoles/length=0.75cm, l_={${R_1}$}, n=R1] (V_{in});
      \draw[-] (D) to (out);
      \draw[-] (D) to (S2);
      \draw[-] (G2) to (3);
      \draw[-] (3) to (4);
      \draw[-] (4) to (5);
      \draw (4.1,3.8) node[nmos, , xscale=1.0, yscale=1.0, rotate=0] (M2) {$M_{2}$};
      \draw (M2.D) -- (D2) (M2.G) -- (G2) (M2.S) -- (S2);
      \draw[-] (S) to (0);
      \draw[-] (D2) to (5);
      \draw[-] (5) to (2);
      \draw[-] (G) to (in);
      \draw (D) node[circ] {};
      \draw (G) node[circ] {};
      \draw (S) node[circ] {};
      \draw (V_{out}) node[ocirc] {};
      \draw (V_{in}) node[ocirc] {};
      \draw (5) node[circ] {};
      \draw (0) node[sground] {};
      \draw (2) node[rground, rotate=180] {};
      \draw[anchor=west] (4.3,6.1) node {$V_{\mathrm{DD}}$};
      \draw[anchor=south east] (out) node {out};
      \draw[anchor=south east] (V_{out}) node {V$_{\mathrm{{out}}}$};
      \draw[anchor=south east] (in) node {in};
      \draw[anchor=south east] (V_{in}) node {V$_{\mathrm{{in}}}$};
    \end{tikzpicture}


  \end{center}
Consider body effect and channel-length modulation, the equivalent resistance at point $out$ look into $M2$: 
\begin{equation}
  R_{out,up} = \frac{1}{g_{m2} + g_{mb2}} || r_o
\end{equation}
With negligible channel-length modulation, we have the low frequency gain:
\begin{equation}
  A_v^0 = -g_{m1} \frac{1}{g_{m2} + g_{mb2}} = - \frac{g_{m1}}{g_{m2}} \frac{1}{1 + \eta}
\end{equation}
where:
\begin{equation}
\frac{1}{1 + \eta} = \frac{g_{mb2}}{g_{m2}}
\end{equation}
Since $I_{DS}$ is same  at $M1$ and $M2$ (from drain to source), we have:
\begin{equation}
  A_v^0 = - \sqrt{\frac{W_1 / L_1}{W_2 / L_2} } \frac{1}{1 + \eta}
  \end{equation} 
  If we replace nmos $M2$ with pmos, the body effect of $M2$ will disappear:
  \begin{equation}
    A_v^0 = - \sqrt{\frac{\mu_n  W_1 / L_1}{\mu_p W_2 / L_2} }
    \end{equation} 
                          \end{minipage}
                };
                %------------ EXEMPLO BASICO BOX ---------------------
                \node[fancytitle, right=10pt] at (box.north west) {Single Stage Amplifier: Common Source (CS), with Diode-Connected Load};
                \end{tikzpicture}



    
                
                \begin{tikzpicture}


            
                  \node [mybox] (box){%
                      \begin{minipage}{0.3\textwidth}
    $M2$ works as a current source:
                        \begin{center}


                          \begin{tikzpicture}[scale=1.00, transform shape, /tikz/circuitikz/bipoles/length=1.50cm, american currents, american voltages, voltage dir=RP]
                            \coordinate (D) at (4.1,2.4);
                            \coordinate (G) at (3,1.4);
                            \coordinate (S) at (4.1,0.4);
                            \coordinate (out) at (5.1,2.4);
                            \coordinate (V_{out}) at (6.1,2.4);
                            \coordinate (G2) at (3,3.8);
                            \coordinate (V_{b}) at (2,3.8);
                            \coordinate (in) at (2,1.4);
                            \coordinate (V_{in}) at (0,1.4);
                            \coordinate (S2) at (4.1,2.8);
                            \coordinate (D2) at (4.1,4.8);
                            \coordinate (0) at (4.1,0);
                            \coordinate (5) at (4.1,5.2);
                            \coordinate (2) at (4.1,5.6);
                           \ctikzset{tripoles/mos style/arrows}
                            \draw (4.1,1.4) node[nmos, , xscale=1.0, yscale=1.0, rotate=0] (M1) {$M_{1}$};
                            \draw (M1.D) -- (D) (M1.G) -- (G) (M1.S) -- (S);
                            \draw[-] (out) to (V_{out});
                            \draw[-] (G2) to (V_{b});
                            \draw (in) to [R, bipoles/length=0.75cm, l_={${R_1}$}, n=R1] (V_{in});
                            \draw[-] (D) to (out);
                            \draw[-] (D) to (S2);
                            \draw (4.1,3.8) node[pmos, , xscale=1.0, yscale=1.0, rotate=0] (M2) {$M_{2}$};
                            \draw (M2.D) -- (S2) (M2.G) -- (G2) (M2.S) -- (D2);
                            \draw[-] (S) to (0);
                            \draw[-] (D2) to (5);
                            \draw[-] (5) to (2);
                            \draw[-] (G) to (in);
                            \draw (D) node[circ] {};
                            \draw (G) node[circ] {};
                            \draw (S) node[circ] {};
                            \draw (V_{out}) node[ocirc] {};
                            \draw (V_{b}) node[ocirc] {};
                            \draw (V_{in}) node[ocirc] {};
                            \draw (0) node[sground] {};
                            \draw (2) node[rground, rotate=180] {};
                            \draw[anchor=west] (4.3,6.1) node {$V_{\mathrm{DD}}$};
                            \draw[anchor=south east] (out) node {out};
                            \draw[anchor=south east] (V_{out}) node {V$_{\mathrm{{out}}}$};
                            \draw[anchor=south east] (G2) node {G2};
                            \draw[anchor=south east] (V_{b}) node {V$_{\mathrm{{b}}}$};
                            \draw[anchor=south east] (in) node {in};
                            \draw[anchor=south east] (V_{in}) node {V$_{\mathrm{{in}}}$};
                          \end{tikzpicture}
                        
                        
                        \end{center}
The low frequency gain:                    
                        \begin{equation}
                          A_v^0 = - g_{m1}(r_{o1} || r_{o2})
                          \end{equation} 

      
                            \end{minipage}
                  };
                  %------------ EXEMPLO BASICO BOX ---------------------
                  \node[fancytitle, right=10pt] at (box.north west) {Single Stage Amplifier: Common Source (CS), with Current-Source Load};
                  \end{tikzpicture}
    

                  \begin{tikzpicture}


            
                    \node [mybox] (box){%
                        \begin{minipage}{0.3\textwidth}
      
      
\begin{center}

  \begin{tikzpicture}[scale=1.00, transform shape, /tikz/circuitikz/bipoles/length=1.50cm, american currents, american voltages, voltage dir=RP]
    \coordinate (D) at (4.1,2.4);
    \coordinate (G) at (3,1.4);
    \coordinate (S) at (4.1,0.4);
    \coordinate (out) at (5.1,2.4);
    \coordinate (V_{out}) at (6.1,2.4);
    \coordinate (in) at (2,1.4);
    \coordinate (V_{in}) at (0,1.4);
    \coordinate (S2) at (4.1,2.8);
    \coordinate (G2) at (2,3.8);
    \coordinate (D2) at (4.1,4.8);
    \coordinate (0) at (4.1,0);
    \coordinate (5) at (4.1,5.2);
    \coordinate (2) at (4.1,5.6);
   \ctikzset{tripoles/mos style/arrows}
    \draw (4.1,1.4) node[nmos, , xscale=1.0, yscale=1.0, rotate=0] (M1) {$M_{1}$};
    \draw (M1.D) -- (D) (M1.G) -- (G) (M1.S) -- (S);
    \draw[-] (out) to (V_{out});
    \draw (in) to [R, bipoles/length=0.75cm, l_={${R_1}$}, n=R1] (V_{in});
    \draw[-] (D) to (out);
    \draw[-] (D) to (S2);
    \draw (4.1,3.8) node[pmos, , xscale=1.0, yscale=1.0, rotate=0] (M2) {$M_{2}$};
    \draw (M2.D) -- (S2) (M2.G) -- (G2) (M2.S) -- (D2);
    \draw[-] (G2) to (in);
    \draw[-] (S) to (0);
    \draw[-] (D2) to (5);
    \draw[-] (5) to (2);
    \draw[-] (G) to (in);
    \draw (D) node[circ] {};
    \draw (G) node[circ] {};
    \draw (S) node[circ] {};
    \draw (V_{out}) node[ocirc] {};
    \draw (in) node[circ] {};
    \draw (V_{in}) node[ocirc] {};
    \draw (0) node[sground] {};
    \draw (2) node[rground, rotate=180] {};
    \draw[anchor=west] (4.3,6.1) node {$V_{\mathrm{DD}}$};
    \draw[anchor=south east] (out) node {out};
    \draw[anchor=south east] (V_{out}) node {V$_{\mathrm{{out}}}$};
    \draw[anchor=south east] (in) node {in};
    \draw[anchor=south east] (V_{in}) node {V$_{\mathrm{{in}}}$};
  \end{tikzpicture}

\end{center}
The low frequency gain:                    
                        \begin{equation}
                          A_v^0 = - (g_{m1} + g_{m2} )(r_{o1} || r_{o2})
                          \end{equation} 
        
                              \end{minipage}
                    };
                    %------------ EXEMPLO BASICO BOX ---------------------
                    \node[fancytitle, right=10pt] at (box.north west) {Single Stage Amplifier: Common Source (CS), with Active Load};
                    \end{tikzpicture}



                    \begin{tikzpicture}


            
                      \node [mybox] (box){%
                          \begin{minipage}{0.3\textwidth}
        

                            \begin{center}
                            \begin{tikzpicture}[scale=1.00, transform shape, /tikz/circuitikz/bipoles/length=1.50cm, american currents, american voltages, voltage dir=RP]
                              \coordinate (D) at (4.1,3.4);
                              \coordinate (G) at (3,2.4);
                              \coordinate (S) at (4.1,1.4);
                              \coordinate (V_{out}) at (5.1,1.4);
                              \coordinate (in) at (2,2.4);
                              \coordinate (V_{in}) at (0,2.4);
                              \coordinate (3) at (4.1,0.4);
                              \coordinate (0) at (4.1,0);
                              \coordinate (2) at (4.1,3.6);
                             \ctikzset{tripoles/mos style/arrows}
                              \draw (4.1,2.4) node[nmos, , xscale=1.0, yscale=1.0, rotate=0] (M1) {$M_{1}$};
                              \draw (M1.D) -- (D) (M1.G) -- (G) (M1.S) -- (S);
                              \draw[-] (S) to (V_{out});
                              \draw (in) to [R, bipoles/length=0.75cm, l_={${R_1}$}, n=R1] (V_{in});
                              \draw (S) to [R, bipoles/length=0.75cm, l_={${R_S}$}, n=RS] (3);
                              \draw[-] (3) to (0);
                              \draw[-] (D) to (2);
                              \draw[-] (G) to (in);
                              \draw (D) node[circ] {};
                              \draw (G) node[circ] {};
                              \draw (S) node[circ] {};
                              \draw (V_{out}) node[ocirc] {};
                              \draw (V_{in}) node[ocirc] {};
                              \draw (0) node[sground] {};
                              \draw (2) node[rground, rotate=180] {};
                              \draw[anchor=west] (4.3,4.1) node {$V_{\mathrm{DD}}$};
                              \draw[anchor=south east] (V_{out}) node {V$_{\mathrm{{out}}}$};
                              \draw[anchor=south east] (in) node {in};
                              \draw[anchor=south east] (V_{in}) node {V$_{\mathrm{{in}}}$};
                            \end{tikzpicture}
                          \end{center}
                          The low frequency gain:                    
                          \begin{equation}
                            A_v^0 =  \frac{g_m R_S}{1 + (g_{m} + g_{mb})R_S}
                            \end{equation} 
        The equivalent resistance at point $out$ look up into $M1$:
        \begin{equation}
          R_{out, up} =  \frac{1}{g_m + g_{mb}} || r_o
          \end{equation} 

          Consider high-frequency gain with $3$ capacitors:
\begin{center}

  \begin{tikzpicture}[scale=1.00, transform shape, /tikz/circuitikz/bipoles/length=1.50cm, american currents, american voltages, voltage dir=RP]
    \coordinate (D) at (5,3.8);
    \coordinate (G) at (3,2.8);
    \coordinate (S) at (5,1.8);
    \coordinate (out) at (6,1.8);
    \coordinate (V_{out}) at (7,1.8);
    \coordinate (in) at (2,2.8);
    \coordinate (V_{in}) at (0,2.8);
    \coordinate (3) at (5,0.8);
    \coordinate (0) at (5,0.4);
    \coordinate (2) at (5,4);
    \coordinate (6) at (3,3.8);
    \coordinate (7) at (4,3.8);
    \coordinate (8) at (3,1.8);
    \coordinate (9) at (4,1.8);
    \coordinate (10) at (6,1.4);
    \coordinate (11) at (6,0.4);
    \coordinate (12) at (6,0);
   \ctikzset{tripoles/mos style/arrows}
    \draw (5,2.8) node[nmos, , xscale=1.0, yscale=1.0, rotate=0] (M1) {$M_{1}$};
    \draw (M1.D) -- (D) (M1.G) -- (G) (M1.S) -- (S);
    \draw[-] (S) to (out);
    \draw[-] (out) to (V_{out});
    \draw (in) to [R, bipoles/length=0.75cm, l_={${R_1}$}, n=R1] (V_{in});
    \draw (S) to [R, bipoles/length=0.75cm, l_={${R_S}$}, n=RS] (3);
    \draw[-] (3) to (0);
    \draw[-] (D) to (2);
    \draw[-] (G) to (in);
    \draw[-] (G) to (6);
    \draw[color=red] (6) to [C, bipoles/length=0.75cm, l_={${C_{\mu}}$}, n=C1] (7);
    \draw[-] (7) to (D);
    \draw[-] (G) to (8);
    \draw[color=red] (8) to [C, bipoles/length=0.75cm, l_={${C_{\pi}}$}, n=C2] (9);
    \draw[-] (9) to (S);
    \draw[-] (out) to (10);
    \draw[color=red] (10) to [C, bipoles/length=0.75cm, l_={${C_{L}}$}, n=C3] (11);
    \draw[-] (11) to (12);
    \draw (D) node[circ] {};
    \draw (G) node[circ] {};
    \draw (S) node[circ] {};
    \draw (out) node[circ] {};
    \draw (V_{out}) node[ocirc] {};
    \draw (V_{in}) node[ocirc] {};
    \draw (0) node[sground] {};
    \draw (2) node[rground, rotate=180] {};
    \draw[anchor=west] (5.2,4.5) node {$V_{\mathrm{DD}}$};
    \draw (12) node[sground] {};
    \draw[anchor=south east] (out) node {out};
    \draw[anchor=south east] (V_{out}) node {V$_{\mathrm{{out}}}$};
    \draw[anchor=south east] (in) node {in};
    \draw[anchor=south east] (V_{in}) node {V$_{\mathrm{{in}}}$};
  \end{tikzpicture}


\end{center}
We have:
\begin{equation}
\tau_{\mu}^{0} = C_{\mu} R_1
\end{equation}
\begin{equation}
  \tau_{\pi}^{0} = C_{\pi} \frac{R_1 + R_S}{1 + g_m R_S}
  \end{equation}
  \begin{equation}
    \tau_{L}^{0} = C_L (r_m || R_S)
    \end{equation}
    \begin{equation}
      \tau_{L}^{\pi} = C_L (R_1 || R_S)
      \end{equation}
      \begin{equation}
        \tau_{\mu}^{L} = C_L R_1
        \end{equation}    
        \begin{equation}
      H^{\mu} = H^{L} = 0
          \end{equation}
\begin{equation}
            H^{\pi} = \frac{R_S}{R_1 + R_S}
\end{equation}
\begin{equation}
  b_1 = C_{\mu}R_1 + C_{\pi} \frac{R_1 + R_S}{1 + g_m R_S} + C_L (r_m || R_S)
\end{equation}
\begin{equation}
  a_1 = C_{\pi} \frac{1}{1 + g_m R_S}
\end{equation} 
\begin{equation}
  b_2 =  \frac{R_1 R_S}{1 + g_m R_S}(C_{\pi} C_{\mu} + C_{\pi} C_L) + C_{L} C_{\mu} (r_m || R_S) R_1
\end{equation}
\begin{equation}
a_2 = b_3 = a_3 = 0
\end{equation}
We ignore body effect:
\begin{equation}
  a_0 =  \frac{g_m R_S}{1 + g_m R_S}
\end{equation} 
Now we have the high frequency voltage gain in $s$ domain:
\begin{equation}
A_v(s) = \frac{g_m R_S + C_{\pi}s}{1 + g_m R_S + [(C_{\mu}g_m R_1 + C_{\pi} + C_L)R_S + (C_{\mu} + C_{\pi})R_1]s + R_1 R_S \Delta C^2 s^2} 
\end{equation}
where,
\begin{equation}
  \Delta C^2 = C_{\mu} C_{\pi} + C_{\pi} C_L +   C_{\mu} C_L
\end{equation}
        \end{minipage}
                      };
                      %------------ EXEMPLO BASICO BOX ---------------------
                      \node[fancytitle, right=10pt] at (box.north west) {Single Stage Amplifier: Common Drain (CD) or Source Follower};
                      \end{tikzpicture}


                      \begin{tikzpicture}


            
                        \node [mybox] (box){%
                            \begin{minipage}{0.3\textwidth}
          
                              \begin{center}

                                \begin{tikzpicture}[scale=1.00, transform shape, /tikz/circuitikz/bipoles/length=1.50cm, american currents, american voltages, voltage dir=RP]
                                  
                              
                                  \coordinate (D) at (4.1,2.4);
                                  \coordinate (G) at (3,1.4);
                                  \coordinate (S) at (4.1,0.4);
                                  \coordinate (out) at (5.1,2.4);
                                  \coordinate (V_{out}) at (6.1,2.4);
                                  \coordinate (V_{b}) at (2,1.4);
                                  \coordinate (in) at (2,3.8);
                                  \coordinate (V_{in}) at (0,3.8);
                                  \coordinate (S2) at (4.1,2.8);
                                  \coordinate (D2) at (4.1,4.8);
                                  \coordinate (G2) at (3,3.8);
                                  \coordinate (0) at (4.1,0);
                                  \coordinate (5) at (4.1,5.2);
                                  \coordinate (2) at (4.1,5.6);
                                 \ctikzset{tripoles/mos style/arrows}
                                  \draw (4.1,1.4) node[nmos, , xscale=1.0, yscale=1.0, rotate=0] (M2) {$M_{2}$};
                                  \draw (M2.D) -- (D) (M2.G) -- (G) (M2.S) -- (S);
                                  \draw[-] (out) to (V_{out});
                                  \draw[-] (G) to (V_{b});
                                  \draw (in) to [R, bipoles/length=0.75cm, l_={${R_1}$}, n=R1] (V_{in});
                                  \draw[-] (D) to (out);
                                  \draw[-] (D) to (S2);
                                  \draw (4.1,3.8) node[nmos, , xscale=1.0, yscale=1.0, rotate=0] (M1) {$M_{1}$};
                                  \draw (M1.D) -- (D2) (M1.G) -- (G2) (M1.S) -- (S2);
                                  \draw[-] (S) to (0);
                                  \draw[-] (D2) to (5);
                                  \draw[-] (5) to (2);
                                  \draw[-] (G2) to (in);
                                  \draw (D) node[circ] {};
                                  \draw (G) node[circ] {};
                                  \draw (S) node[circ] {};
                                  \draw (V_{out}) node[ocirc] {};
                                  \draw (V_{b}) node[ocirc] {};
                                  \draw (V_{in}) node[ocirc] {};
                                  \draw (0) node[sground] {};
                                  \draw (2) node[rground, rotate=180] {};
                                  \draw[anchor=west] (4.3,6.1) node {$V_{\mathrm{DD}}$};
                                  \draw[anchor=south east] (out) node {out};
                                  \draw[anchor=south east] (V_{out}) node {V$_{\mathrm{{out}}}$};
                                  \draw[anchor=south east] (V_{b}) node {V$_{\mathrm{{b}}}$};
                                  \draw[anchor=south east] (in) node {in};
                                  \draw[anchor=south east] (V_{in}) node {V$_{\mathrm{{in}}}$};
                              
                                \end{tikzpicture}
                              
                              \end{center}
                              The low frequency gain:                    
                              \begin{equation}
                                A_v^0 =  \frac{R_{eq}}{R_{eq} + \frac{1}{g_{m1}}}
                                \end{equation} 
where:
\begin{equation}
  R_{eq} = \frac{1}{g_{mb1}}|| r_{o1} || r_{o2}
  \end{equation}
            
                                  \end{minipage}
                        };
                        %------------ EXEMPLO BASICO BOX ---------------------
                        \node[fancytitle, right=10pt] at (box.north west) {Single Stage Amplifier: Common Drain (CD) or Source Follower, With Transistor Load};
                        \end{tikzpicture}



                        \begin{tikzpicture}


            
                          \node [mybox] (box){%
                              \begin{minipage}{0.3\textwidth}
            $M_2$ works as current source, $C_1$ is  coupled capacitor:
\begin{center}
  \begin{tikzpicture}[scale=1.00, transform shape, /tikz/circuitikz/bipoles/length=1.50cm, american currents, american voltages, voltage dir=RP]
    \coordinate (D) at (3,2.4);
    \coordinate (G) at (1.9,1.4);
    \coordinate (S) at (3,0.4);
    \coordinate (in) at (1,2.4);
    \coordinate (V_{in}) at (0,2.4);
    \coordinate (G2) at (1.9,3.8);
    \coordinate (V_{b1}) at (0.9,3.8);
    \coordinate (V_{b2}) at (0.9,1.4);
    \coordinate (out) at (3,4.8);
    \coordinate (V_{out}) at (4,4.8);
    \coordinate (0) at (2,2.4);
    \coordinate (S2) at (3,2.8);
    \coordinate (5) at (3,5.2);
    \coordinate (6) at (3,6.2);
    \coordinate (2) at (3,6.6);
    \coordinate (0_split0) at (3,0);
   \ctikzset{tripoles/mos style/arrows}
    \draw (3,1.4) node[nmos, , xscale=1.0, yscale=1.0, rotate=0] (M2) {$M_{2}$};
    \draw (M2.D) -- (D) (M2.G) -- (G) (M2.S) -- (S);
    \draw[-] (in) to (V_{in});
    \draw[-] (G2) to (V_{b1});
    \draw[-] (G) to (V_{b2});
    \draw[-] (out) to (V_{out});
    \draw[-] (D) to (0);
    \draw[-] (D) to (S2);
    \draw (0) to [C, l_=$C_{1}$, n=C1] (in);
    \draw (3,3.8) node[nmos, , xscale=1.0, yscale=1.0, rotate=0] (M1) {$M_{1}$};
    \draw (M1.D) -- (out) (M1.G) -- (G2) (M1.S) -- (S2);
    \draw[-] (S) to (0_split0);
    \draw[-] (out) to (5);
    \draw (5) to [R, bipoles/length=0.75cm, l_=$R_{D}$, n=RD] (6);
    \draw[-] (6) to (2);
    \draw (D) node[circ] {};
    \draw (G) node[circ] {};
    \draw (S) node[circ] {};
    \draw (V_{in}) node[ocirc] {};
    \draw (V_{b1}) node[ocirc] {};
    \draw (V_{b2}) node[ocirc] {};
    \draw (out) node[circ] {};
    \draw (V_{out}) node[ocirc] {};
    \draw (0_split0) node[sground] {};
    \draw (2) node[rground, rotate=180] {};
    \draw[anchor=west] (3.2,7.1) node {$V_{\mathrm{DD}}$};
    \draw[anchor=south east] (in) node {in};
    \draw[anchor=south east] (V_{in}) node {V$_{\mathrm{{in}}}$};
    \draw[anchor=south east] (G2) node {G2};
    \draw[anchor=south east] (V_{b1}) node {V$_{\mathrm{{b1}}}$};
    \draw[anchor=south east] (V_{b2}) node {V$_{\mathrm{{b2}}}$};
    \draw[anchor=south east] (out) node {out};
    \draw[anchor=south east] (V_{out}) node {V$_{\mathrm{{out}}}$};
  \end{tikzpicture}

\end{center}
The low frequency voltage gain:
\begin{equation}
  A_v^0 = g_m(1+\eta)R_D
\end{equation}
              
                                    \end{minipage}
                          };
                          %------------ EXEMPLO BASICO BOX ---------------------
                          \node[fancytitle, right=10pt] at (box.north west) {Single Stage Amplifier: Common Gate (CG)};
                          \end{tikzpicture}
            


                          \begin{tikzpicture}


            
                            \node [mybox] (box){%
                                \begin{minipage}{0.3\textwidth}
            $M_3$ and $M_4$ are formed the cascode current source:
                                  \begin{center}
                \begin{tikzpicture}[scale=0.8, transform shape, /tikz/circuitikz/bipoles/length=1.50cm, american currents, american voltages, voltage dir=RP]
                
                  \coordinate (D) at (2.1,2.4);
                  \coordinate (G) at (1,1.4);
                  \coordinate (S) at (2.1,0.4);
                  \coordinate (G2) at (1,3.8);
                  \coordinate (V_{b2}) at (0,3.8);
                  \coordinate (V_{in}) at (0,1.4);
                  \coordinate (G3) at (1,6.8);
                  \coordinate (V_{b3}) at (0,6.8);
                  \coordinate (G4) at (1,8.8);
                  \coordinate (V_{b4}) at (0,8.8);
                  \coordinate (out) at (2.1,4.8);
                  \coordinate (V_{out}) at (3.1,4.8);
                  \coordinate (S2) at (2.1,2.8);
                  \coordinate (0) at (2.1,0);
                  \coordinate (5) at (2.1,5.8);
                  \coordinate (6) at (2.1,7.8);
                  \coordinate (7) at (2.1,9.8);
                  \coordinate (2) at (2.1,10.8);
                 \ctikzset{tripoles/mos style/arrows}
                  \draw (2.1,1.4) node[nmos, , xscale=1.0, yscale=1.0, rotate=0] (M1) {$M_{1}$};
                  \draw (M1.D) -- (D) (M1.G) -- (G) (M1.S) -- (S);
                  \draw[-] (G2) to (V_{b2});
                  \draw[-] (G) to (V_{in});
                  \draw[-] (G3) to (V_{b3});
                  \draw[-] (G4) to (V_{b4});
                  \draw[-] (out) to (V_{out});
                  \draw[-] (D) to (S2);
                  \draw (2.1,3.8) node[nmos, , xscale=1.0, yscale=1.0, rotate=0] (M2) {$M_{2}$};
                  \draw (M2.D) -- (out) (M2.G) -- (G2) (M2.S) -- (S2);
                  \draw[-] (S) to (0);
                  \draw[-] (out) to (5);
                  \draw (2.1,6.8) node[pmos, , xscale=1.0, yscale=1.0, rotate=0] (M3) {$M_{3}$};
                  \draw (M3.D) -- (5) (M3.G) -- (G3) (M3.S) -- (6);
                  \draw (2.1,8.8) node[pmos, , xscale=1.0, yscale=1.0, rotate=0] (M4) {$M_{4}$};
                  \draw (M4.D) -- (6) (M4.G) -- (G4) (M4.S) -- (7);
                  \draw[-] (7) to (2);
                  \draw (D) node[circ] {};
                  \draw (G) node[circ] {};
                  \draw (S) node[circ] {};
                  \draw (V_{b2}) node[ocirc] {};
                  \draw (V_{in}) node[ocirc] {};
                  \draw (V_{b3}) node[ocirc] {};
                  \draw (V_{b4}) node[ocirc] {};
                  \draw (out) node[circ] {};
                  \draw (V_{out}) node[ocirc] {};
                  \draw (0) node[sground] {};
                  \draw (2) node[rground, rotate=180] {};
                  \draw[anchor=west] (2.3,11.3) node {$V_{\mathrm{DD}}$};
                  \draw[anchor=south east] (G2) node {G2};
                  \draw[anchor=south east] (V_{b2}) node {V$_{\mathrm{{b2}}}$};
                  \draw[anchor=south east] (V_{in}) node {V$_{\mathrm{{in}}}$};
                  \draw[anchor=south east] (G3) node {G3};
                  \draw[anchor=south east] (V_{b3}) node {V$_{\mathrm{{b3}}}$};
                  \draw[anchor=south east] (G4) node {G4};
                  \draw[anchor=south east] (V_{b4}) node {V$_{\mathrm{{b4}}}$};
                  \draw[anchor=south east] (out) node {out};
                  \draw[anchor=south east] (V_{out}) node {V$_{\mathrm{{out}}}$};


                \end{tikzpicture}
              \end{center}
              The low frequency voltage gain:
              \begin{equation}
                A_v^0 \approx -g_{m1} ( g_{m2}r_{o1} r_{o2} || g_{m3}r_{o3}r_{o4})
              \end{equation}
                
                                      \end{minipage}
                            };
                            %------------ EXEMPLO BASICO BOX ---------------------
                            \node[fancytitle, right=10pt] at (box.north west) {Single Stage Amplifier: Cascode};
                            \end{tikzpicture}
              




            \begin{tikzpicture}


            
              \node [mybox] (box){%
                  \begin{minipage}{0.3\textwidth}

  
                        \end{minipage}
              };
              %------------ EXEMPLO BASICO BOX ---------------------
              \node[fancytitle, right=10pt] at (box.north west) {Single Stage Amplifier: Common Drain (CD) or Source Follower};
              \end{tikzpicture}


\end{multicols*}
\end{document}
