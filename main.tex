%%%%%%%%%%%%%%%%%%%%%%%%%%%%%%%%%%%%%%%%%%%%%%%%%%%%%%%
% MatPlotLib and Random Cheat Sheet
%
% Edited by Michelle Cristina de Sousa Baltazar
%
% http://matplotlib.org/api/pyplot_summary.html
% http://matplotlib.org/users/pyplot_tutorial.html
%
%%%%%%%%%%%%%%%%%%%%%%%%%%%%%%%%%%%%%%%%%%%%%%%%%%%%%%%

\documentclass{article}
\usepackage[landscape]{geometry}
\usepackage{url}
\usepackage{multicol}
\usepackage{amsmath}
\usepackage{amsfonts}
\usepackage{tikz}
\usetikzlibrary{decorations.pathmorphing}
\usepackage{amsmath,amssymb}

\usepackage{colortbl}
\usepackage{xcolor}
\usepackage{mathtools}
\usepackage{amsmath,amssymb}
\usepackage{enumitem}


\usepackage{geometry}
 \geometry{
  a3paper
 }


\usepackage[utf8]{inputenc}

\advance\topmargin-.8in
\advance\textheight3in
\advance\textwidth3in
\advance\oddsidemargin-1.5in
\advance\evensidemargin-1.5in
\parindent0pt
\parskip2pt
\newcommand{\hr}{\centerline{\rule{3.5in}{1pt}}}
%\colorbox[HTML]{e4e4e4}{\makebox[\textwidth-2\fboxsep][l]{texto}
\begin{document}

\begin{center}{\huge{\textbf{Analog CMOS Integrated Circuit Design Cheat Sheet}}}\\
{\large By Xiao Ma (mxlol233@outlook.com)}
\end{center}
\begin{multicols*}{3}

\tikzstyle{mybox} = [draw=black, fill=white, very thick,
    rectangle, rounded corners, inner sep=10pt, inner ysep=10pt]
\tikzstyle{fancytitle} =[fill=black, text=white, font=\bfseries]
%------------ CONTEÚDO CAIXA RANDOM ---------------
\begin{tikzpicture}
\node [mybox] (box){%
    \begin{minipage}{0.3\textwidth}
		Process parameters ($n$,$V_{TH}$,$KP$, $V_E$):
      \begin{equation}
        t_{OX} = \frac{L_{min}}{50}
       \end{equation}


       \begin{equation}
        t_{si} = \sqrt{\frac{2\epsilon_{si}(\Phi - V_{BD})}{qN_B}}
       \end{equation}

       \begin{equation}
        C_{OX} = \frac{\bold{\epsilon}_{OX}}{t_{OX}}
       \end{equation}

       \begin{equation}
        C_{D} = \frac{\bold{\epsilon}_{si}}{t_{si}}
       \end{equation}

       \begin{equation}
        KP = \mu C_{OX}
       \end{equation}



      \begin{equation}
       \bold{\beta} = KP\frac{W}{L}
      \end{equation}


\begin{equation}
  Q_{dep} = \sqrt{4q\epsilon_{si}|\Phi_F|N_{sub}}
\end{equation}


\begin{equation}
  V_{TH0} = \Phi_{MS} + 2 \Phi_F + \frac{Q_{dep}}{C_{OX}}
\end{equation}


      \begin{equation}
        V_{TH} = V_{TH0} + \gamma (\sqrt{|2\Phi_F| + V_{BS}} - \sqrt{|2\Phi_F|})
       \end{equation}

       \begin{equation}
        n = \frac{\gamma}{\sqrt{|2\Phi_F| + V_{BS}}} = 1 + \frac{C_D}{C_{OX}}
       \end{equation}
In linear region:
\begin{equation}
  I_{DS} = \beta[(V_{GS}-V_{TH})V_{DS} - \frac{1}{2}V_{DS}^2]
\end{equation}

\begin{equation}
  R_{on} = \frac{1}{\beta (V_{GS} - V_{TH})}
\end{equation}

      Channel-Length modulation in saturation region:

      \begin{equation}
        K' = \frac{KP}{2n}
      \end{equation}

      \begin{equation}
        \lambda = \frac{1}{V_E L}
      \end{equation}
      \begin{equation}
       I_{DS} =  K' \frac{W}{L}(V_{GS}-V_{TH})^2(1+\lambda V_{DS})
      \end{equation}
      \begin{equation}
        r_{o} = \frac{\partial V_{DS}}{\partial I_{DS}} \approx \frac{1}{\lambda I_{DS}} = \frac{V_E L}{I_{DS}}
      \end{equation}
Saturation region has three distinctive regions: weak-inversion (exponential region),  
strong-inversion, and velocity saturation.
 


    \end{minipage}
};
%------------ CAIXA RANDOM ---------------------
\node[fancytitle, right=10pt] at (box.north west) {Model of MOS Transistors};
\end{tikzpicture}

%------------ CONTEUDO EXEMPLO BASICO ---------------------
\begin{tikzpicture}
  \node [mybox] (box){%
      \begin{minipage}{0.3\textwidth}
\begin{center}
        \small{\begin{tabular}{lp{1.4cm} l}
          {\bf Names} & {\bf Symbols} & {\bf Values} \\
          dielectric constant of sub-silicon & $\epsilon_{si}$ & $1$ pF/cm \\ \hline
          dielectric constant of gate-oxide & $\epsilon_{OX}$ & $0.34$ pF/cm \\ \hline
          electron charge & $q$ & $1.6 \times 10^{-19}$ C \\ \hline
        minium channel length & $L_{min}$ & $0.35$ $\mu$m \\ \hline
        width of gate-oxide  & $t_{OX}$ & $0.1$ nm \\ \hline
        width of depletion layer & $t_{si}$ & $7$ nm \\ \hline
        junction built-in voltage & $\Phi$ & $0.6$ V\\ \hline
        drain-bulk voltage & $V_{BD}$ & $-3.3 $V \\ \hline
         gate-oxide capacitance & $C_{OX}$ & $0.5$ $\mu$F/cm$^2$ \\ \hline
         depletion layer capacitance & $C_{D}$ & $0.1$ $\mu$F/cm$^2$ \\ \hline
         bulk doping level & $N_B$ & $4\times10^{17}$ cm$^{-3}$ \\ \hline
         P type mobility rate & $\mu_{p}$& $250$ cm$^2$/Vs\\ \hline
         N type mobility rate & $\mu_{n}$&  $600$ cm$^2$/Vs \\ \hline
         N type KP & $KP_{n}$ &  $300$ $\mu$A/V$^2$ \\ \hline
         & $n$ &  $1.2 \cdots 1.5$ \\ \hline
         & $|2\Phi_F|$ &  $0.6$ V \\ \hline
         & $\gamma$ &  $0.5 \cdots 0.8$ V$^{\frac{1}{2}}$ \\ \hline
         N type $K'$ & $K'_{n}$ &  $100$ $\mu$A/V$^2$ \\ \hline
         P type $K'$ & $K'_{p}$ &  $40$ $\mu$A/V$^2$ \\ \hline
           & $V_{GSTt}$ &  $70$ mV \\ \hline
           & $v_{sat}$ &  $10^{7}$ cm/s \\ \hline
           & $\theta L$ &  $0.2$ $\mu$m/V \\ \hline
           
            \end{tabular}}

            
          \end{center}

            \end{minipage}
  };
  %------------ EXEMPLO BASICO BOX ---------------------
  \node[fancytitle, right=10pt] at (box.north west) {Value Examples In $0.35\mu m$ Process Nodes };
  \end{tikzpicture}


  \begin{tikzpicture}
    \node [mybox] (box){%
        \begin{minipage}{0.3\textwidth}

         \begin{equation}
          I_{DS}= I_{D0}\frac{W}{L}e^{\frac{V_{GS}}{n\frac{KT}{q}}}
         \end{equation}

  
\begin{equation}
g_{m} = \frac{\partial I_{DS}}{\partial V_{GS}} = \frac{I_{DS}}{n\frac{KT}{q}}
\end{equation}



   
  
              \end{minipage}
    };
    \node[fancytitle, right=10pt] at (box.north west) {Weak-Inversion};
    \end{tikzpicture}

    \begin{tikzpicture}
      \node [mybox] (box){%
          \begin{minipage}{0.3\textwidth}
Ignore channel-length modulation:
            \begin{equation}
              I_{DS} =  K' \frac{W}{L}(V_{GS}-V_{TH})^2
             \end{equation}

             \begin{equation}
              g_m = \frac{2I_{DS}}{V_{GS} - V_{TH}}
             \end{equation}
         

                \end{minipage}
      };
      %------------ EXEMPLO BASICO BOX ---------------------
      \node[fancytitle, right=10pt] at (box.north west) {strong-inversion};
      \end{tikzpicture}


    
        \begin{tikzpicture}
          \node [mybox] (box){%
              \begin{minipage}{0.3\textwidth}
                The voltage  and current  at transition point between weak-inversion and strong-inversion:
                \begin{equation}
                 V_{GSt} =  2n\frac{KT}{q} + V_{TH} 
                \end{equation}
                \begin{equation}
                 I_{DSt} \approx  K'\frac{W}{L} (2n\frac{KT}{q})^2  
                \end{equation}
          EKV model, a smooth model for weak-inversion and strong-inversion regions:
   
          \begin{equation}
           I_{DS} =  K' \frac{W}{L}(V_{GS}-V_{TH})^2[ ln(1+e^{\frac{V_{GS}}{V_{GSt}}}) ]^2
          \end{equation}
          Let:
          \begin{equation}
           v =  \frac{V_{GS}}{V_{GSt}}
          \end{equation}
          \begin{equation}
           i =  \frac{I_{DS}}{I_{DSt}} = [ ln(1+e^{v}) ]^2
          \end{equation}
          then,
          \begin{equation}
            v  = ln(e^{\sqrt{i}} -1)
           \end{equation}
           \begin{equation}
            V_{GS} - V_{TH}  = V_{GSTt}ln(e^{\sqrt{i}} -1)
           \end{equation}
where:
\begin{equation}
  V_{GSTt} = V_{GSt} - V_{TH} = 2n\frac{KT}{q} 
\end{equation}
          When $v=1$, $i=1$, we also have:
          \begin{equation}
           I_{DSt} = K'\frac{W}{L}(V_{GSt} - V_{TH})^2
          \end{equation}
        
                    \end{minipage}
          };
          %------------ EXEMPLO BASICO BOX ---------------------
          \node[fancytitle, right=10pt] at (box.north west) {Transition Point Between Weak-Inversion and Strong-Inversion};
          \end{tikzpicture}
    

              
        \begin{tikzpicture}
          \node [mybox] (box){%
              \begin{minipage}{0.3\textwidth}
    
                \begin{equation}
                  I_{DS} =  WC_{OX}v_{sat}(V_{GS}-V_{TH})
                 \end{equation}
                 \begin{equation}
                  g_m=  WC_{OX}v_{sat}
                 \end{equation}

        
                    \end{minipage}
          };
          %------------ EXEMPLO BASICO BOX ---------------------
          \node[fancytitle, right=10pt] at (box.north west) {Velocity Saturation};
          \end{tikzpicture}
    

              
        \begin{tikzpicture}
          \node [mybox] (box){%
              \begin{minipage}{0.3\textwidth}

               

                A smooth model for strong-inversion  and velocity saturation regions:
                \begin{equation}
                  I_{DS} = \frac{K' \frac{W}{L}(V_{GS}-V_{TH})^2}{1 + \theta (V_{GS} - V_{TH})} 
                 \end{equation}
        
                 where:
        \begin{equation}
          \theta = \frac{\mu}{2n}\frac{1}{v_{sat}L}
         \end{equation}
         $\theta L$ is constant:
         \begin{equation}
          \theta L = \frac{\mu}{2n}\frac{1}{v_{sat}}
         \end{equation}

         \begin{equation}
          g_{m,sat} =WC_{OX}v_{sat}= \frac{K'W}{\theta L}
         \end{equation}


         The voltage  and current  at transition point between strong-inversion  and velocity saturation:
         \begin{equation}
          V_{GSt} = \frac{1}{\theta} + V_{TH} = 2nL\frac{v_{sat}}{\mu} + V_{TH}
         \end{equation}
         \begin{equation}
          I_{DSt} = K'WL(2n\frac{v_{sat}}{\mu})^2
         \end{equation}


                    \end{minipage}
          };
          %------------ EXEMPLO BASICO BOX ---------------------
          \node[fancytitle, right=10pt] at (box.north west) {Transition Point Between Strong-Inversion and Velocity Saturation};
          \end{tikzpicture}
    
      
          \begin{tikzpicture}
            \node [mybox] (box){%
                \begin{minipage}{0.3\textwidth}
  
                 
  AAAAAAAAAAAAAAAAAAAAAAAAAAA
  AAAAAAAAAAAAAAAAAAAAAAAAAAA
  AAAAAAAAAAAAAAAAAAAAAAAAAAA
  AAAAAAAAAAAAAAAAAAAAAAAAAAA
  AAAAAAAAAAAAAAAAAAAAAAAAAAA
  AAAAAAAAAAAAAAAAAAAAAAAAAAA

  
                      \end{minipage}
            };
            %------------ EXEMPLO BASICO BOX ---------------------
            \node[fancytitle, right=10pt] at (box.north west) {Operational-Amp: Biasing Circuits};
            \end{tikzpicture}
      

\end{multicols*}
\end{document}
